\documentclass[12pt, a4paper, utf8]{book}

\usepackage[ngerman]{babel}
\usepackage[hidelinks]{hyperref}
\usepackage{etoolbox}

\title{x86-Assembly \\ \small{amd64 unter Linux}}
\author{github.com/khde}
\date{\today}


\begin{document}
\maketitle

\tableofcontents{}

\chapter*{Vorwort}

\part{Grundlagen}


\part{Einstieg in Assembly}
\chapter{Aufbau}
\section{Struktur}
\subsection{.text Sektion}
\subsection{.data Sektion}
\subsection{.bss Sektion}
\subsection{Weitere Sektionen}
\section{Syntax}
    Intel vs AT\&T
    Opcode, Mnemonic
\section{Adressiermoduse}
    Register, Speicher, Direkt
\section{Erstes Programm}
    Hallo Welt
% \chapter{Anweisungen}
\chapter{Datenübertragung}
\section{mov}
% \section{movsx, movsxd}
% \section{movzx}
\section{lea}
\section{xchg}
\chapter{Ganzzahlige Arithmetik}
\section{Addieren}
\subsection{add}
\subsection{adc}
\subsection{xadd}
\section{Subtrahieren}
\subsection{sub}
\subsection{sbb}
\section{Multiplizieren}
\subsection{mul}
\subsection{imul}
\section{Dividieren}
\subsection{div}
\subsection{idiv}
\section{Inkrement inc}
\section{Dekrement dec}
\chapter{Logik}
\section{and}
\section{or}
\section{not}
\section{xor}
\section{test}
\chapter{Bedingtes Ausführen}
\section{Springen jmp}
\section{Bedingtes Springen cmp, jCC}
\section{rflags-Register}
    jz, jnz, je, jne, jg, jge, jl, jle, ja, jae, jb, jbe, jo, js, jecxz
% \section{cmpsb, cmpsw, cmpsd, cmpsq, cmpxchg}
\section{loop, loopne}
% \section{cmove, cmovl, cmova}
\section{rflags Manipulieren}
\subsection{Carry-Flag}
    stc, clc, cmc
\subsection{Direction-Flag}
    std, cld
\subsection{Interrupt-Flag}
    sti, cli
\section{lahf, sahf}
\chapter{Bit Operationen}
\section{Bit Shifting}
\subsection{Logischer Shift: shr, shl}
\subsection{Arithmetischer Shift: sar, sal}
\subsection{shld, shrd}
\section{Bit Rotation}
\subsection{ror}
\subsection{rol}
\subsection{rcr}
\subsection{rcl}
\chapter{Bit Manipulation}
\section{bts}
\section{btr}
\section{bt}
\section{bswap}
    setCC
\chapter{Stack}
\section{Grundlagen des Stacks}
\section{push, pop}
\section{Die Rolle von rsp, rbp}
% \section{pushad, pusha, pushfd, pushf, popad, popa, popfd, popf}
\section{Stack Frame und Stack alignment}
    push rbp
    mov rbp, rsp
\chapter{Funktionen}
\section{Funktionen mit call und ret}
\section{Funktionsprolog und Funktionsepilog}
    enter, leave
\section{Aufrufskonventionen}
    System V AMD64 ABI
\section{Registerinhalt sichern}
    -caller, callee saved\\
    -push eax\\
     equivalent zu\\
     sub esp, 4\\
     mov [esp], eax\\
    \\
    -pop eax\\
        sub esp, 4\\
        mov [esp], eax\\
\chapter{Systemaufrufe}
    -Kernel im Ring 0 (Kernel-Modus)
    -Zugriff auf gesamten CPU-Befehlssatz und Speicherbereich
    -Normaler Prozess läuft im unprivilegierten Ringen 1 - 3 (Benutzer-Modus)
\section{syscall, int 0x80}
\chapter{Zeichenketten und Felder}
\section{stosb/stosw/stosd}
\section{movsb, rep stosb}
\chapter{Kommandozeilenargumente}
    int main(int argc, char *argv[])
                 rdi         rsi
\chapter{Weitere Befehle}
\section{nop}
    nop
    xchg rax, rax
    mov rax, rax
\section{ASCII und BCD}
\subsection{aaa}
\subsection{aas}
\subsection{aad}
\subsection{aam}
\subsection{daa}
\subsection{das}
    -Nicht Vorhanden in 64-Bit Modus

\appendix
\chapter{Wichtige Befehle}
\chapter{Wichtige Linux-Syscalls}
\chapter{Register}
\chapter{Erweiterungen}
\chapter{ASCII-Tabelle}

\end{document}
